\documentclass{article}
\usepackage[utf8]{inputenc}

\usepackage[T2A]{fontenc}
\usepackage[utf8]{inputenc}
\usepackage[russian]{babel}

\usepackage{minted}

\title{Алгоритмы и алгоритмические языки}
\author{Лисид Лаконский}
\date{February 2023}

\begin{document}
\raggedright

\maketitle
\tableofcontents
\pagebreak

\section{Алгоритмы и алгоритмические языки - 07.02.2023}

\subsection{Лабораторная работа №1}

Методические указания по выполнению лабораторной работы №1: https://github.com/BFI-2202/algorithms\_laboratories

\hfill

Лабораторные работы могут выполняться как в одном документе, так и в отдельных документов — титульные листы не должны нумероваться

\hfill

Первые шестнадцать человек принадлежат \textbf{Гурикову Сергею Ростиславовичу}, остальные принадлежат \textbf{Загвоздкиной Анне Викторовне}

\hfill

Все изображения нужно оформлять по ГОСТу — перед ними \textbf{делать на них ссылки}

Примеры \textbf{неправильного} описания изображения: «Рисунок 1 — метод Vvod()», пример \textbf{правильного}: «Рисунок 1 — метод, предназначенный для конвертации строковых значений, считанных из формы ввода, в числовые»


\end{document}
