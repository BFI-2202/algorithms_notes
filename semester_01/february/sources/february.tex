\documentclass{article}
\usepackage[utf8]{inputenc}

\usepackage[T2A]{fontenc}
\usepackage[utf8]{inputenc}
\usepackage[russian]{babel}

\usepackage{minted}

\title{Алгоритмы и алгоритмические языки}
\author{Лисид Лаконский}
\date{February 2023}

\begin{document}
\raggedright

\maketitle
\tableofcontents
\pagebreak

\section{Алгоритмы и алгоритмические языки - 01.02.2023}

dll — \textbf{динамически подключаемая библиотека}

Для создания dll в Visual Studio необходимо выбрать шаблон «\textbf{библиотека классов}»

\subsection{Проект «вычисление суммы квадратов двух чисел»}

\subsubsection{Динамическая библиотека}

Для решения поставленной задачи нам необходимо реализовать: \textbf{метод нахождения суммы квадратов двух чисел, метод для ввода данных, метод для вывода данных}

\begin{minted}{csharp}
using System;
using System.Windows.Forms;

namespace ClassLibrary1
{
    public class Class1 {
        public static int Vvod(TextBox t)
        {
            return Convert.ToInt(t.Text);
        }
        public static int Vyvod(TextBox t, int c)
        {
            t.Text = Convert.ToString(c);
        }
        public static int Sum_kv(int x, int y)
        {
            int res = x * x + y * y;
            return res;
        }
    }
}
\end{minted}

В настоящих проектах рекомендуется давать классам, методам, переменным и так далее \textbf{нормальные имена}, а не как в коде выше.

\subsubsection{Основная программа}

Помещаем в форму с помощью вкладки «элементы управления» следующие компоненты: три надписи (\textbf{label}), два поля ввода текста (\textbf{textbox}) и одну кнопку (\textbf{button})

Код собственно программы:

\begin{minted}{csharp}
using System;
// Прочие юзинги...
using LibraryClass1;

namespace ClassProgram1 {
    public partial class Form1 : Form
    {
        public Form1()
        {
            InitializeComponent();
        }
        private void button1_Click(object sender, EventArgs e)
        {
            int a = Class1.Vvod(textBox1);
            int b = Class1.Vvod(textBox2);
            int y = Class1.Sum_kv(a, b);
            Class1.Vyvod(textBox3, y);
        }
    }
}
\end{minted}

\end{document}